% Chapter 5

\chapter{跨分辨率遥感影像计数实验及分析}
\section{实验设计}
本设计探究的是跨分辨率遥感影像的计数问题,在跨分辨率车辆计数数据集上,训练本文设计的网络,测试其性能。数据集中共包含了拍摄于同一位置不同时间的192张192 张极低分辨率图像和 8 张高分辨率图像,低分辨率图像中包含8张和高分辨率图像再同一天拍摄的图像。如何通过少量的高分辨率图像的计数结果,得到可以在低分辨率图像上做出准确计数估计的模型就是本设计的目标。本设计基于CRVC-Net骨架网络,通过独特设计的3个注意力门,更好的综合了从高分辨率及其余两个时间的低分辨率图像的特征信息,逐层与解码器输出进行计算,改进输出效果。

本方法先得到从低分辨率图像得到的分割图,然后据此估计出覆盖率,再由覆盖率通过回归模型估计出最总计数结果,因此主要从以下三个维度度量模型性能:
\begin{enumerate}    
    \item 测试模型在低分辨率(LR)图像上的分割结果,与相应的高分辨率(HR)图像的真实标注进行比较。
    \item 在所有低分辨率图像上测试的车辆覆盖率结果,与人工标注进行比较。
    \item 测试模型在低分辨率图像上的计数结果,与相应高分辨率图像的真实标注进行比较。
\end{enumerate}
\section{分割精度测试}
以同一日期的高分辨率图像的分割图作为标注,本设计在各种网络模块的组合上测试了分割精度。采用像素精度(Pixel Accuracy,PA)来定量评估分割精度。它表示所有像素中正确分类的比例。
\begin{equation}
    P A=\frac{\sum_{i}^{k} p_{i i}}{\sum_{i}^{k} \sum_{j}^{k} p_{i j}}
\end{equation}
假设有k+1个类别(包括k个目标类别和1个背景类别),pij表示属于类别i但被预测为类别j的像素数量。具体来说,pii代表真正例(TP),pij代表假正例(FP),pji代表假负例(FN)。
\begin{enumerate}    
    \item 仅使用CRVC-Net骨架进行训练。
    \item 仅使用简单注意力门的CRVC-Net进行训练。
    \item 仅使用自注意力门的CRVC-Net进行训练。
    \item 仅使用跨分辨率注意力门的CRVC-Net进行训练。
    \item 仅使用时间序列注意力门的CRVC-Net进行训练。
    \item 使用自注意力门和跨分辨率注意力门的CRVC-Net进行训练。
    \item 使用自注意力门和时间序列注意力门的CRVC-Net进行训练。
    \item 使用自注意力门、时间序列注意力门和跨分辨率注意力门的CRVC-Net进行训练。
\end{enumerate}

\subsection{消融实验}
为了验证并解释模块设计的合理性和因果性,针对3种注意力门设计了消融实验。
对于跨分辨率图像计数问题,我们有少量的高分辨率图像和同日同区域的低分辨率图像,同时还包括距离该日期较近和较远的两张低分辨率图像作为模型的输入。对于训练的CRVC-Net这样的U-Net网络而言,跳跃连接会接受来自主分支的编码器的中间结果,而没有使用CRSC分支和IRTC分支的中间结果。本文设计的网络将来自主分支、CRSC分支和IRTC分支的编码器的中间结果均作为解码器跳跃连接的输入。

\section{测试}
在训练阶段,四张图像输入到网络中。第一输入是ILR,低分辨率图像用于在骨架中生成车辆分割。第二输入是相应的高分辨率图像IHR,起到空间引导的作用。ILR和IHR应该在8对相应的低分辨率和高分辨率图像中选取。我们使用其中的7对进行训练,1对进行测试。第三和第四输入是一对低分辨率图像,即ILR close和ILR far,以强调时间连续性。ILR close选为与ILR日期最接近的图像,而ILR far在图像集中随机选取,与ILR的时间差超过50天。由于低分辨率和高分辨率图像对数量较少,为了增强鲁棒性,将训练图像裁剪为32×32的补丁后,通过旋转90度、180度和270度,上下左右翻转,将训练集扩增至4000个图像补丁。测试集扩增到1000个图像补丁。在解码器的上采样过程中,使用U-Net结构将上采样的图像与相应比例的特征图连接。网络输出与原始图像补丁大小相同的32×32的最终分割图。由于训练集是通过旋转和翻转方法生成的,最终的分割结果应该是来自同一图像的各种增强图像的平均值。然后将32×32的补丁拼接成所需的图像大小,以获取整个停车场区域的图像。在测试阶段,只需将ILR输入到分割骨架中,输出即为车辆分割图。考虑到起重机的数量太少,对分割结果影响不大,我们排除了这一类别,训练和测试阶段只涵
\section{训练细节}
超参数
loss选择

