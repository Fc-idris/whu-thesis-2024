% Chapter 6

\chapter{总结与展望}
\section{结论}
本文设计了一种基于多头注意力机制的U-Net网络模型,同时使用焦点损失平衡类别数目不平衡的问题。本设计测试了该方法在跨分辨率遥感影像车辆计数问题上的准确性,尤其是对于稀少类别的估计准确率有不小的提升。模型通过实现不同的注意力门(自注意力门、跨分辨率注意力门、时间序列注意力门),展现了模型在处理复杂图像数据时的灵活性。通过设计消融实验,验证了各注意力门的重要性和影响,显示单独和组合使用各注意力门对模型性能的不同影响。最终实现的三种注意力门组合提供了最高的像素精度,说明了多头的注意力机制门的在综合时间连续性和空间一致性上的有效性。
\section{不足与展望}

本模型虽然在CRVC数据集上取得了不错的计数估计结果,但也仍存在一些不足之处。多头注意力机制虽然增强了模型的性能,但也大幅增加了参数数量,这导致更高的内存需求和可能的存在过拟合问题。高参数量和复杂的结构可能引起优化过程中的问题,如梯度消失或梯度爆炸,这要求更精细的参数调整和优化策略。同时CRVC数据集较小的数据量可能限制了模型的泛化能力,对于未见过的更大或更复杂的数据集,模型可能需要进一步的调整和优化。

本研究表明,虽然设计的模型在跨分辨率图像的车辆计数任务中表现出了良好的性能,但对于未来的工作,探索参数效率更高的模型架构、增强模型的泛化能力以及优化内存和计算资源的使用,将是进一步提升模型应用实用性的关键方向。进一步发掘高低分辨率图像间的隐含关系也将影响模型预测性能。对于其他涉及到跨分辨率图像问题,该模型也具有很强的迁移潜力。

