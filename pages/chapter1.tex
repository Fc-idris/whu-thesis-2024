% Chapter 1

\chapter{绪论}
随着遥感技术的快速发展,我们能获得到越来越丰富的遥感观测数据。既有高分辨率的清晰图像,也包括低分辨率的鸟瞰图。这些图像从更宏观的视角为我们提供了许多有价值的数据。然而如何合理的使用这些图像数据,特别是从低分辨率图像中找出更多难以通过人类肉眼直接识别出的有效信息成为了一个重要的研究课题。在车辆计数等真实场景中,高分辨率图像虽然能提供丰富的细节,但有着获取成本高昂以及难以获取稳定连续的高频率数据的问题。本文在CRVC跨分辨率车辆计数数据集上,设计了一种基于注意力机制的深度学习U型网络模型,旨在提高跨分辨率遥感图像中隐含的空间和时间信息进行更丰富全面的表示,以进一步提升目标计数的准确性。
\section{研究背景与意义}
\subsection{高分辨率和低分辨率遥感图像简介}
在遥感领域,分辨率是用来描述遥感图像细节程度的一个重要指标,根据遥感卫星搭载的传感器不同,可以从多个维度进行描述,包括空间分辨率、时间分辨率、光谱分辨率和辐射分辨率。在讨论高分辨率(High Resolution, HR)图像和低分辨率(Low Resolution, LR)图像时,通常指的是空间分辨率的差异。空间分辨率的详细定义可以参照\cite{MultispectralImageAnalysisUsingObjectOrientedParadigm}:
(i)低或粗分辨率定义为地面采样距离(GSD)为 30 m 或更大的像素,(ii)中分辨率是 GSD范围为 2.0–30 m,(iii) 高分辨率为 GSD 0.5–2.0 m,以及 (iv) 极高分辨率为像素大小 <0.5 m

本文中将该定义简化,只进行高分辨率与低分辨图像的区分。
\subsubsection{高分辨率图像}
高分辨率图像通常指具有较高空间分辨率的图像,即图像中单个像素所代表的地面面积较小,能够显示更加精细的地面特征。高分辨率图像使得用户可以观察到较小的地面对象,例如单个车辆、道路标线甚至是行人。虽然“高分辨率”这个术语没有绝对的定义,但在遥感领域,我们把空间分辨率小于1米(通常在0.3米到1米之间)的图像常被认为是高分辨率图像。目前我们可以使用的高分辨率遥感图像来源主要有:航空摄影(搭载高分辨率摄像机或低空高分辨率无人机拍摄的数据)和某些高性能的卫星遥感仪器,例如WorldView系列、GeoEye-1、QuickBird等。
高分辨率图像中的精细地面特征信息,在城市规划、交通监控、农业监测(如作物健康分析)、详细的地物分类、灾害评估等领域都有广泛的使用。特别是在目标识别领域中,高分辨率的图像可以使用深度学习中多种模型和方法,具有很高的应用价值。


\subsubsection{低分辨率图像}

低分辨率图像指的是空间分辨率较低的图像,即图像中单个像素所代表的地面面积较大,只能显示较为粗糙的地面特征。低分辨率图像难以分辨较小的地面对象,但适合于观察大范围的地表变化。通常空间分辨率大于10米(如10米、30米或更大)的图像被认为是低分辨率图像。低分辨率图像主要来源于具有宽幅覆盖能力的卫星遥感仪器,如MODIS(具有数百米的空间分辨率)、Landsat系列(15米到30米分辨率)、Sentinel-2(10米到60米分辨率)等。图像中的大范围地表特征,在气候变化研究、大范围土地覆盖变化监测、环境监测、城市发展规划、海洋和大气研究等领域有着很高的应用价值。对于目标识别以及目标计数领域来说,模糊的图像质量大大加大了现有深度学习模型的识别难度。


\subsubsection{重访周期与成本}
地球卫星的轨道半径和周期可以根据开普勒第三定律和牛顿的万有引力定律推导出来的。下面是公式推导的过程:
牛顿的万有引力定律描述了两个物体之间的引力,公式为:
\begin{equation}
    F = G \frac{m_1 m_2}{r^2}​​
\end{equation}
其中,F是两个物体之间的引力,G是万有引力常数,$m_1​$和$m_2$​是两个物体的质量,r是两个物体的中心之间的距离。
对于在圆轨道上运动的卫星,向心加速度$a_c$提供了必要的向心力,可由下式给出:
\begin{equation}
    F_c = m \frac{v^2}{r}​​​
\end{equation}
这里,$F_c$​是向心力,m是卫星的质量,v是卫星的轨道速度,r是卫星绕行天体的轨道半径。
由于卫星在轨道上的向心力正是由地球对卫星的万有引力提供的,可得到:
\begin{equation}
    G \frac{M m}{r^2} = m \frac{v^2}{r}​​​​
\end{equation}
轨道周期T是卫星完成一圈轨道所需的时间,与轨道速度和轨道半径相关,可表示为:
\begin{equation}
    T = \frac{2\pi r}{v}​​​​
\end{equation}
在一般情况下,当考虑两个物体质量都不能忽略时,M + m代替上式中的M,最终得到两个物体之间的轨道周期T与轨道半径r之间的关系:
\begin{equation}
    T = 2\pi \sqrt{\frac{r^3}{G (M + m)}}​​​​
\end{equation}
这就是天体绕另一天体旋转周期的计算公式,它揭示了轨道周期与轨道半径之间的关系。这个公式在天文学和航天工程中有广泛应用,用于预测天体运动及设计人造卫星的轨道。
高分辨率图像一般由低轨卫星拍摄,具有较小的轨道半径和周期。同时也因此具有较小的视场(the field of view, FOV)较小。高分辨率图像辨率卫星通常使用任务驱动模式进行地球观测。这意味着,如果不提前提交观测任务,一颗卫星需要6个月才能完成全球覆盖,获得特定区域的图像的重访周期相当长。此外,高分辨率图像非常昂贵,例如 WorldView-3 的价格为 34 美元/km2。
作为对比,低分辨率卫星的往往运行在更高的轨道上,虽然空间分辨率有所降低,但视场较大。因此获得同一地点重访周期要短得多,例如 PlanetScope 卫星每天重访一次,价格也低得多,为 1.8 美元/平方公里。

\subsection{研究意义}
单独依靠低分辨率图像进行目标计数是十分困难的,而仅通过高分辨率图像进行目标计数,不仅花费巨大,同时还需要面对连续监控数据的缺失。如何通过低成本且具有时间连续性的低分辨率图像进行目标计数及实时监测就成为解决问题的关键。本文提出的方法通过少量高分辨率图像的辅助,在时间连续的低分辨率图像上实现目标计数,具有很高的应用价值。该方法不仅局限于目标计数,更好地利用了低分辨率图像中蕴含的模糊信息,在稠密车流人流识别、智慧城市设计等领域也有很高的应用潜力。

\section{国内外研究现状}
\subsection{跨分辨率遥感影像目标计数现状}
高分辨率图像计数主要有两类方法。一类是检测方法,通过识别出具体的物体位置来进一步计数。另一种是计数类方法,直接估计图像中对应物体的数目。
\subsubsection{目标检测}
基于检测的目标计数是目标检测下的一个分支。目标检测作为计算机视觉的一个主要研究方向,主要研究的是如何识别出特定的对象,并确定它的大小和位置。目前随着深度学习技术的发展,利用神经网络进行目标检测已经成为主流方案。主要的目标检测算法包括RCNN系列、SSD、YOLO系列等
\subsubsection{语义分割}
语义分割是一类特殊的目标检测任务,需要给出每个像素的分类。主要的语义分割模型包括全卷积网络(FCN)、U-Net、SegNet、DeepLab 系列、Mask R-CNN和Attention U-Net。
\subsubsection{基于回归的计数方法}
这类方法从深度神经网络提取的特征中直接进行数目的回归估计,常常用于处理目标密集、相互遮挡严重的场景,如人群计数。
\subsubsection{基于密度图的计数方法}
基于密度图的计数方法通过估计目标区域的目标密度,从而计算出数量估计。目前主要的密度图估计方法包括CSRNet、MCNN和TEDNet。
\subsubsection{跨分辨率车辆计数数据集}
该数据集有少量高分辨率图像和拍摄于这些高分辨率图像日期之间的相同位置的低分辨率图像组成。该数据集主要关注低分辨率下的此车辆计数问题。在低分辨率下,汽车的边界极难分辨,人类在标注识别的过程中也并不容易。该数据集提供了少量拍摄了同位置的高分辨率图像作为参照。目前,已经有一些工作在这个数据集上取得了不错的效果。在本文的第三部分将进行进一步说明。
\section{研究目标与内容}
\subsection{研究目标}
\begin{enumerate}    
    \item 跨分辨率车辆计数算法的开发:开发一种新的基于深度学习的车辆计数算法,该算法能够有效利用有限的高分辨率图像来指导低分辨率图像中的车辆计数。
    \item 探究高效利用数据集中空间一致性和时间连续性信息的方式:探究同一时刻高分辨率与低分辨率图像间的空间一致性和连续低分辨率图像间的时间连续性的高效利用方式。研究不同分辨率图像对算法效果的具体影响。
    \item 探究注意力机制在跨分辨率目标计数的应用:探索注意力机制在提高跨分辨率车辆计数准确性中的应用,尤其是如何通过注意力机制来增强模型综合多种特征表示的能力。
\end{enumerate}

\subsection{研究内容}
\begin{enumerate}
    \item 研究背景与意义分析:分析相关遥感技术的发展背景,高分辨率与低分辨率遥感图像的特点及其在车辆计数中的面临困难和挑战。梳理了当前目标检测、语义分割、基于回归和密度图的计数方法等方面的研究进展,特别关注跨分辨率图像处理及车辆计数领域的最新研究成果。
    \item 跨分辨率车辆计数数据集分析:对CRVC数据集进行分析调研,了解其数据组成分布以及数据中隐含的性质的分析及建模。
    \item 基于注意力机制的车辆计数模型设计:设计并实现一种新的基于注意力机制的深度学习模型,用于提高跨分辨率遥感图像车辆计数的准确性。
    \item 算法的性能评估及优化:在CRVC数据集上评估设计算法的性能,进行消融实验,并与现有的车辆计数方法进行比较。进一步优化算法,以达到更高的计数准确性和更好的泛化能力。
\end{enumerate}

\section{章节安排}
第一章为绪论,整体介绍研究背景及研究内容。第二章为目标计数方法计数框架,详细解释了目标计数领域的方法细节。第三章为跨分辨率车辆计数数据集介绍,包括了数据的组成、分布以及对数据性质的初步分析。第四章介绍了基于注意力机制的跨分辨率遥感影像计数方法,详细解释了基于注意力机制的深度学习网络设计细节。
