% Chapter 4

\chapter{基于注意力机制的跨分辨率遥感影像计数}
\section{}
\section{代码}
\section{代码}
\section{代码}
\subsection{原始代码}
朴实的代码块:

使用 verbatim 环境可以得到如下原样的输出。
\begin{verbatim}
print("Hello world!")
\end{verbatim}

使用 listings 包提供的 lstlisting 环境可以对代码进行进一步的格式化,minted 包所提供的 minted 环境还可以对代码进行高亮。更多定制功能请自行参照文档配置。

\subsection{算法描述/伪代码}
参考 \href{https://en.wikibooks.org/wiki/LaTeX/Algorithms}{Algorithms} 与 algorithm2e 文档,给出一个简单的示例,见算法 \ref{alg:alg1}。

\begin{algorithm}
  \SetAlgoLined
  \KwData{this text}
  \KwResult{how to write algorithm with \LaTeXe}
  initialization\;
  \While{not at end of this document}{
    read current\;
    \eIf{understand}{
      go to next section\;
      current section becomes this one\;
    }{
      go back to the beginning of current section\;
    }
  }
  \caption{如何写算法}\label{alg:alg1}
\end{algorithm}

\section{绘图}
关于使用 \LaTeX{} 绘图的更多例子,请参考 \href{https://www.overleaf.com/learn/latex/Pgfplots_package}{Pgfplots package}。一般建议使用如 Photoshop、PowerPoint 等制图,再转换成 PDF 等格式插入。

\section{写在最后}
工具不重要,对工具的合理运用才重要。希望本模板对大家的论文写作有所帮助。
