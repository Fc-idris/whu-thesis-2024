% Chapter 2

\chapter{目标计数方法技术框架}
\section{目标计数}
目前目标计数领域主要有三类方法。一类是检测方法,通过目标检测模型识别出具体的物体位置,之后根据结果来进一步计数。但这类方法对于输入图像的分辨率有着较高的要求,往往需要物体具有明确清晰的边缘特征。在低分辨率下往往表现效果较差。一种是基于回归的方法,直接拟合出图像特征和目标数目之间的回归模型得到图像中对应物体的数目。但这种方法未能完整利用图像中的空间,及序列信息。当输入图像的大小和分布有变化的情况下,往往不具有很强的泛化能力。另一类方法是基于密度图的目标计数方法。此类方法通常先得出一个目标物体在区域内的一个分部,之后就可以通过密度分布来估计总体的数量。该方法在稠密计数的场景下往往具有较好的效果。在本文使用的跨分辨率车辆计数数据集上,可以把车辆计数视为一个稠密计数场景。使用基于密度图的计数方法相较其余两类方法有着更好的表现。
受上述方法启发,本文将跨分辨率车辆计数问题转换为两个子问题,即综合跨分辨率图像信息的图像分割网络和映射分割结果和最终计数目标的回归模型。
\section{语义分割模型}
语义分割作为计算机视觉的一个核心研究方向,目前已经有了较为成熟的解决方法。它的目标是对图像中的每个像素进行细致的分类,从而实现对图像的像素级理解。像素级的输出能力使得该领域的很多方法在密度图的估计上也有着不错的表现。
\subsection{卷积层}
对于图像数据,常使用卷积层而不是全连接层来进行特征提取。卷积层具有的平移不变性和局部性非常适合处理图像数据,可以掌握图像的空间特征。下面给出卷积层的基本定义。
\subsubsection{卷积模型}
\begin{equation}
  \label{eq:eq_conv-layer}
  [\mathbf{H}]{i, j} = u + \sum_{a = -\Delta}^{\Delta} \sum_{b = -\Delta}^{\Delta} [\mathbf{V }]{a, b} [\mathbf{X}]_{i+a, j+b}
\end{equation}
通过使用系数$[\mathbf{V}]{a, b}$对位置$(i, j)$附近的像素$(i+a, j+b)$进行加权得到$[\mathbf{H}]{i, j}$。 其中$|a|> \Delta$或$|b| > \Delta$约束条件使得该式满足局部性,即只关注于在位置像素$(i+a, j+b)$的小领域范围内的参数,大大减少了参数量。$\mathbf{V}$被称为卷积核(convolution kernel)或者滤波器(filter),也是该卷积层的权重,通常该权重是可学习的参数。参数a,b也对应着卷积核的尺寸$k_h,k_w$。
\subsubsection{多通道输入与多通道输出}
上式\eqref{eq:eq_conv-layer}是单通道情况下卷积层的数学表示,当输入图像的通道数为$c_i$时,那么我们需要构造一个形状为$c_i\times k_h\times k_w$的卷积核。由于输入和卷积核都有$c_i$个通道,我们可以对每个通道输入的二维张量和卷积核的二维张量进行互相关运算,再对通道求和得到一个二维张量。这就是一个输出通道的结果。如果我们需要输出通道数为$c_o$时,只需创建一个卷积核的形状是$c_o\times c_i\times k_h\times k_w$。通道数量可以视作对于不同特征的描述,随着神经网络层数的加深,通常的做法是减少空间分辨率的同时增加通道数量。
\subsubsection{填充和步幅}
在应用多层卷积时,我们常常丢失边缘像素。填充(padding)可以解决这个问题。在输入图像的边界填充一定数量的元素(通常填充元素是$0$)。  
通常,如果我们添加$p_h$行填充(大约一半在顶部,一半在底部)和$p_w$列填充(左侧大约一半,右侧一半),则输出形状将为

\begin{equation}
  (n_h-k_h+p_h+1),(n_w-k_w+p_w+1)
\end{equation}
这意味着输出的高度和宽度将分别增加$p_h$和$p_w$。
在许多情况下,我们可以设置$p_h=k_h-1$和$p_w=k_w-1$,这样使得输入和输出具有相同的高度和宽度。假设$k_h$是奇数,我们将在高度的两侧填充$p_h/2$行。 如果$k_h$是偶数,通常会在输入顶部填充$\lceil p_h/2\rceil$行,在底部填充$\lfloor p_h/2\rfloor$行。同理,我们填充宽度的两侧。

感受野是指卷积网络中某一层输出特征图上的一个元素所对应的输入图像上的区域大小。它表征着特征图能“看到”的区域的大小。我们可以通过连续的卷积来增加感受野,但这会增加参数量。我们还可以通过调整步幅来增大感受野。
步幅是卷积操作中卷积核移动的步长。在对图像进行卷积时,卷积核从图像的一个角落开始,按照指定的步幅在图像上滑动,每次移动指定的像素数,直到覆盖整个图像。当步幅大于1时,卷积核每次移动多个像素,输出的特征图的尺寸也会随之减小。具体公式如下:

通常,当垂直步幅为$s_h$,水平步幅为$s_w$时,输出形状为
\begin{equation}
  \lfloor(n_h-k_h+p_h+s_h)/s_h\rfloor \times \lfloor(n_w-k_w+p_w+s_w)/s_w\rfloor
\end{equation}
如果我们设置了$p_h=k_h-1$和$p_w=k_w-1$,则输出形状将简化为$\lfloor(n_h+s_h-1)/s_h\rfloor \times \lfloor(n_w+s_w-1)/s_w\rfloor$。 更进一步,如果输入的高度和宽度可以被垂直和水平步幅整除,则输出形状将为$(n_h/s_h) \times (n_w/s_w)$。
\subsection{激活函数}
卷积神经网络中常用的激活函数包括ReLU(线性整流单元)、Sigmoid、Tanh(双曲正切)等。这些激活函数的目的是在网络中引入非线性特性,使得网络能够学习到更加复杂的数据表示。本文用到的是 ReLU (Rectified Linear Unit)函数。
对于给定元素$x$,ReLU函数被定义为该元素与$0$的最大值。它是目前最常用的激活函数之一。因为它的导数在大于0时为1,小于0时为0,这使得它可以用来缓解梯度消失的问题。
\begin{equation}
  f(x) = \max(0, x)
\end{equation}

\subsection{池化层}
池化(pooling)是卷积神经网络中常见的一种方法,主要用于减少特征图的维度,减少计算量的同时保留重要的一致性信息。与卷积层类似,池化运算也是通过一个固定形状的窗口滑动来实现的。与之不同的是,池化通过对邻近像素进行统计学操作(如取最大值或平均值)来实现,因此也不包含参数。主要有两种类型的池化:最大池化(Max Pooling)和平均池化(Average Pooling)。
池化操作通常有两个参数:池化核的大小(KxK)和步幅(S)。池化核指定了池化操作的邻域范围,步幅定义了池化操作的移动间隔。对于输入大小为$W \times H$的特征图,池化操作后的输出大小$W' \times H'$可以通过以下公式计算:
\begin{equation}
  W' = \left\lfloor\frac{W - K}{S} + 1\right\rfloor
\end{equation}
\begin{equation}
  H' = \left\lfloor\frac{H - K}{S} + 1\right\rfloor
\end{equation}

在卷积网络的实践中,池化层通常有降低特征维度、引入不变性、增加鲁棒性和防止过拟合的作用。

\subsection{权重衰减}
在模型训练时,可能会遇到过拟合的问题,使得模型在已有数据上有着较好的性能,而在测试数据上表现不佳。我们可以使用多种正则化技术来缓解过拟合的问题。权重衰减(weight decay)是最广泛使用的正则化的技术之一, 它通常也被称为$L_2$正则化。
$L_2$ 正则化在损失函数中添加模型权重的平方之和作为惩罚项。同时通过一个非负的超参数$\lambda$来控制正则化的强度。$L_2$正则化正则化修正后的损失函数如下式:
\begin{equation}
  L(\mathbf{w}, b)=\frac{1}{n} \sum_{i=1}^{n} \frac{1}{2}\left(\mathbf{w}^{\top} \mathbf{x}^{(i)}+b-y^{(i)}\right)^{2}+\frac{\lambda}{2}\|\mathbf{w}\|^{2}
\end{equation}

L2正则化的目的是鼓励模型学习到更小更分散的权重值,从而提高模型的泛化能力。它对大的权重值施加较大的惩罚,从而防止模型依赖于少数几个可能具有高噪声的特征。

\subsection{暂退法}
Dropout在训练过程中以一定几率随机“丢弃”(即暂时移除)网络中的一部分神经元(包括其连接),这有助于模型学习到更加鲁棒的特征,减少神经元间复杂的共适应关系。需要注意的是,在测试时,我们通常不使用dropout。
\subsection{批量归一化}
批量归一化(Batch Normalization)是通过对每个小批量数据进行归一化处理,调整神经网络中间层的输出,使其均值接近0,标准差接近1。这可以通过减去它们的均值除以它们的标准差得到。这有助于稳定和加速深度网络的训练过程,同时也具有一定的正则化效果。
批量归一化(Batch Normalization,简称BN)是一种在深度神经网络中广泛使用的技术,用于加速训练过程并提高模型的稳定性。其基本思想是在网络的每层之后添加一个归一化步骤,这个步骤会对每个小批量数据(mini-batch)进行归一化处理,以确保网络中间层的激活分布保持稳定。批量归一化的公式如下:

\begin{equation}
  \hat{x}_i = \frac{x_i - \mu_B}{\sqrt{\sigma_B^2 + \epsilon}} 
\end{equation}
其中,$\epsilon$是一个很小的数,用来防止除以零。归一化后的\(\hat{x}_i\)具有零均值和单位方差。



\subsection{损失函数}
\subsection{梯度下降}
\section{U-Net网络}
U-Net是一个广泛被应用的语义分割模型,U-Net是一个具有对称结构的网络,通过使用跳跃连接(Skip Connection)来结合低层次的位置信息和高层次的语义信息,从而在细节上进行更准确的预测。
\section{注意力机制}
\section{Attention UNet网络}
Attention UNet网络
\section{光流估计}

\section{跨分辨率遥感图像目标计数技术路线}
在文中引用公式可以这么写:\(a^2 + b^2 = c^2\)。这是勾股定理,它还可以表示为 \(c = \sqrt{a^2 + b^2}\)。还可以让公式单独一段并且加上编号:
\begin{equation}
  \sin^2{\theta} + \cos^2{\theta} = 1 \label{eq:pingfanghe}
\end{equation}
注意,公式前请不要空行。

还可以通过添加标签在正文中引用公式,如式\eqref{eq:pingfanghe}。

我们还可以轻松打出一个漂亮的矩阵:
\begin{equation}
  \vb*{A} =
  \begin{bmatrix}
    1  & 2  & 3  & 4  \\
    11 & 22 & 33 & 44 \\
  \end{bmatrix} \times
  \begin{bmatrix}
    22 & 24 \\
    32 & 34 \\
    42 & 44 \\
    52 & 54 \\
  \end{bmatrix}
\end{equation}

或者多行对齐的公式:
\begin{equation}
  \begin{aligned}
    f_1(x) & = (x + y)^2         \\
           & = x^2 + 2 x y + y^2
  \end{aligned}
\end{equation}

模板使用了 unicode-math 包更改数学字体。所以在使用数学字体时,尽量使用 unicode-math 包提供的 \verb|\sym| 接口,详情请阅读 unicode-math 文档。

\section{插图的使用}
\begin{figure}
  \centering
  \includegraphics[width=0.3\textwidth]{whulogo.pdf}
  \caption{插图示例}
  \label{fig:whu}
\end{figure}

\LaTeX{} 环境下可以使用常见的图片格式:JPEG、PNG、PDF 等。当然也可以使用 \LaTeX{} 直接绘制矢量图形,可以参考 pgf/ti\emph{k}z 等包中的相关内容。需要注意的是,无论采用什么方式绘制图形,首先考虑的是图片的清晰程度以及图片的可理解性,过于不清晰的图片将可能会浪费很多时间。

\verb|[htbp]| 选项分别是此处、页顶、页底、独立一页。\verb|[width=\textwidth]| 让图片占满整行,或 \verb|[width=2cm]| 直接设置宽度。可以随时在文中进行引用,如图~\ref{fig:whu},建议缩放时保持图像的宽高比不变。

如果一个图由两个或两个以上分图组成时,各分图分别以(a)、(b)、(c)...... 作为图序,并须有分图题。模板使用 subcaption 宏包来处理,比如图~\ref{fig:subfig-a} 和图~\ref{fig:subfig-b}。

\begin{figure}[h]
  \centering
  \begin{subfigure}{0.2\textwidth}
    \includegraphics[width=\linewidth]{whulogo.pdf}
    \caption{武汉大学校徽}
    \label{fig:subfig-a}
  \end{subfigure}\qquad
  \begin{subfigure}{0.7\textwidth}
    \includegraphics[width=\linewidth]{whu.pdf}
    \caption{武汉大学}
    \label{fig:subfig-b}
  \end{subfigure}
  \caption{多个分图的示例}
  \label{fig:multi-image}
\end{figure}

\section{表格的使用}
表格的输入可能会比较麻烦,可以使用在线的工具,如 \href{https://www.tablesgenerator.com/}{Tables Generator} 能便捷地创建表格,也可以使用离线的工具,如 \href{https://ctan.org/pkg/excel2latex}{Excel2LaTeX} 支持从 Excel 表格转换成 \LaTeX{} 表格。\href{https://en.wikibooks.org/wiki/LaTeX/Tables}{LaTeX/Tables} 上及 \href{https://www.tug.org/pracjourn/2007-1/mori/mori.pdf}{Tables in LaTeX} 也有更多的示例能够参考。

\subsection{普通表格}
下面是一些普通表格的示例:

\begin{table}[ht]
  \centering
  \caption{简单表格}
  \label{tab:1}
  \begin{tabular}{|l|c|r|}
    \hline
    我是 & 一只 & 普通 \\
    \hline
    的   & 表格 & 呀   \\
    \hline
  \end{tabular}
\end{table}

也可以使用 booktabs 包创建三线表。

\begin{table}[ht]
  \centering
  \caption{一般三线表}
  \label{tab:2}
  \begin{tabular}{ccc}
    \toprule
    姓名 & 学号 & 性别 \\
    \midrule
    张三 & 001  & 男   \\
    李四 & 002  & 女   \\
    \bottomrule
  \end{tabular}
\end{table}

三线表中三条横线分别使用 \verb|\toprule|、\verb|\midrule| 与 \verb|\bottomrule|。若要添加 \(m\)--\(n\) 列的横线,可使用 \verb|\cmidrule{m-n}| 。

要创建占满给定宽度的表格需要使用到 tabularx 包提供的 tabularx 环境。引用表格与其它引用一样,只需要如表~\ref{tab:3}。

\begin{table}[ht]
  \centering
  \caption{占满文字宽度的三线表}
  \label{tab:3}
  \begin{tabularx}{\textwidth}{CCCC}
    \toprule
    序号 & 年龄 & 身高   & 体重  \\
    \midrule
    1    & 14   & 156    & 42    \\
    2    & 16   & 158    & 45    \\
    3    & 14   & 162    & 48    \\
    4    & 15   & 163    & 50    \\
    \cmidrule{2-4} %添加2-4列的中线
    平均 & 15   & 159.75 & 46.25 \\
    \bottomrule
  \end{tabularx}
\end{table}

\subsection{跨页表格}
跨页表格常用于附录(把正文懒得放下的实验数据统统放在附录的表中)。一般使用 longtable 包提供的 longtable 环境。若要要创建占满给定宽度的跨页表格,可以使用 xltabular 包提供的 xltabular 环境,使用方法与 longtable 类似。以下是一个文字宽度的跨页表格的示例:

\begin{xltabular}{\textwidth}{CCCCCCCCC}
  \caption{文字宽度的跨页表格示例}  \\
  \toprule
  1 & 0 & 5 & 1 & 2 & 3 & 4 & 5 & 6 \\
  \midrule
  \endfirsthead

  \multicolumn{9}{l}{接上一页}      \\
  \toprule
  1 & 0 & 5 & 1 & 2 & 3 & 4 & 5 & 6 \\
  \midrule
  \endhead

  \toprule
  \multicolumn{9}{r}{转下一页}
  \endfoot

  \bottomrule
  \endlastfoot

  1 & 0 & 5 & 1 & 2 & 3 & 4 & 5 & 6 \\
  1 & 0 & 5 & 1 & 2 & 3 & 4 & 5 & 6 \\
  1 & 0 & 5 & 1 & 2 & 3 & 4 & 5 & 6 \\
  1 & 0 & 5 & 1 & 2 & 3 & 4 & 5 & 6 \\
  1 & 0 & 5 & 1 & 2 & 3 & 4 & 5 & 6 \\
  1 & 0 & 5 & 1 & 2 & 3 & 4 & 5 & 6 \\
  1 & 0 & 5 & 1 & 2 & 3 & 4 & 5 & 6 \\
  1 & 0 & 5 & 1 & 2 & 3 & 4 & 5 & 6 \\
  1 & 0 & 5 & 1 & 2 & 3 & 4 & 5 & 6 \\
  1 & 0 & 5 & 1 & 2 & 3 & 4 & 5 & 6 \\
  1 & 0 & 5 & 1 & 2 & 3 & 4 & 5 & 6 \\
  1 & 0 & 5 & 1 & 2 & 3 & 4 & 5 & 6 \\
  1 & 0 & 5 & 1 & 2 & 3 & 4 & 5 & 6 \\
  1 & 0 & 5 & 1 & 2 & 3 & 4 & 5 & 6 \\
  1 & 0 & 5 & 1 & 2 & 3 & 4 & 5 & 6 \\
  1 & 0 & 5 & 1 & 2 & 3 & 4 & 5 & 6 \\
  1 & 0 & 5 & 1 & 2 & 3 & 4 & 5 & 6 \\
  1 & 0 & 5 & 1 & 2 & 3 & 4 & 5 & 6 \\
  1 & 0 & 5 & 1 & 2 & 3 & 4 & 5 & 6 \\
  1 & 0 & 5 & 1 & 2 & 3 & 4 & 5 & 6 \\
\end{xltabular}


\section{列表的使用}
下面演示了创建有序及无序列表,如需其它样式,\href{https://www.latex-tutorial.com/tutorials/lists/}{LaTeX Lists} 上有更多的示例。

\subsection{有序列表}
这是一个计数的列表
\begin{enumerate}
  \item 第一项
        \begin{enumerate}
          \item 第一项中的第一项
          \item 第一项中的第二项
        \end{enumerate}
  \item 第二项
        \begin{enumerate}[label=(\roman*)]
          \item 第一项中的第一项
          \item 第一项中的第二项
        \end{enumerate}
  \item 第三项
\end{enumerate}

\subsection{不计数列表}
这是一个不计数的列表
\begin{itemize}
  \item 第一项
        \begin{itemize}
          \item 第一项中的第一项
          \item 第一项中的第二项
        \end{itemize}
  \item 第二项
  \item 第三项
\end{itemize}

\begin{table}[b]
  \caption{模板定义的数学环境}\label{tab:数学环境}
  \begin{tabularx}{\textwidth}{CCCC}
    \toprule
    theorem     & definition & lemma  & corollary \\
    定理        & 定义       & 引理   & 推论      \\
    \midrule
    proposition & example    & remark & proof     \\
    性质        & 例         & 注     & 证明      \\
    \bottomrule
  \end{tabularx}
\end{table}

\section{数学环境的使用}
模板简单定义了 8 种数学环境,具体见表~\ref{tab:数学环境},使用方法如下所示。

\begin{theorem}
  设向量 \(\vb*{a} \neq \vb*{0}\),那么向量 \(\vb*{b} \parallel \vb*{a}\) 的充分必要条件是:存在唯一的实数 \(\lambda\),使 \(\vb*{b} = \lambda \vb*{a}\)。
\end{theorem}
\begin{definition}
  这是一条定义。
\end{definition}
\begin{lemma}
  这是一条引理。
\end{lemma}
\begin{corollary}
  对数轴上任意一点 \(P\),轴上有向线段 \(\overrightarrow{OP}\) 都可唯一地表示为点 \(P\) 的坐标与轴上单位向量 \(\vb*{e}_u\) 的乘积:\(\overrightarrow{OP} = u \vb*{e}_u\)。
\end{corollary}
\begin{proposition}
  这是一条性质。
\end{proposition}
\begin{example}
  这是一条例。
\end{example}
\begin{remark}
  这是一条注。
\end{remark}
\begin{proof}
  留作练习。
\end{proof}

若要定义自己的数学环境,可通过如下代码实现:
\begin{verbatim}
\newtheorem{nonsense}{胡说}
\newtheorem*{bullshit}{八道}
\end{verbatim}
其中,带星号 * 的命令不会自动编号。

\newtheorem{nonsense}{胡说}
\newtheorem*{bullshit}{八道}

\begin{nonsense}
  啊吧啊吧啊吧。
\end{nonsense}

\begin{bullshit}
  不啦不啦不啦。
\end{bullshit}

\section{单位}
单位的输入请使用 siunitx 包中提供的 \verb|\si| 与 \verb|\SI| 命令,可以方便地处理希腊字母以及数字与单位之间的空白。在以前,\LaTeX{} 中输入角度需要使用 \verb|$^\circ$| 的奇技淫巧,现在只需要 \verb|\ang| 命令解决问题。当然 siunitx 包中还提供了不少其他有用的命令,有需要的可以自行阅读 siunitx 文档。

示例:\SI{6.4e6}{m},\SI{9}{\micro\meter},\si{kg.m.s^{-1}},\ang{104;28;}。

\section{物理符号}
physics 宏包可以让用户更加方便、简洁地使用、输入物理符号,具体也请自行阅读 physics 文档。示例如下
\begin{equation}
  \begin{aligned}
    \int_0^{2\symup{\pi}} \abs{\sin{x}} \dd{x} & = 2 \int_0^{\symup{\pi}} \sin{x} \dd{x} \\
                                                      & = -2 \eval{\cos{x}}_0^{\symup{\pi}}     \\
                                                      & = 4
  \end{aligned}
\end{equation}