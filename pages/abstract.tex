% 中文摘要

随着遥感技术的飞速进步,我们获得了丰富的高分辨率(HR)和低分辨率(LR)遥感图像资源。HR图像能提供丰富的细节信息,然而却面临着成本和连续性等限制。易于获取且成本低廉的LR图像具有很高的应用潜力。然而,如何有效地从LR图像中提取难以直观识别的信息,成为了一个挑战。本文针对这一问题,在CRVC跨分辨率车辆计数数据集上设计并实现了一种基于注意力机制的深度学习U型网络模型,旨在充分利用遥感图像中隐含的空间和时间信息,从而提高目标计数的准确性。

本研究对CRVC数据集中的高分辨率和低分辨率图像之间的空间一致性及时间连续性进行了分析,并设计了时间注意力模块和跨分辨率注意力模块来针对性的处理数据。研究展示了注意力机制在提升模型性能中的应用潜力,尤其是在增强模型对多种特征表示的综合能力的方面。本文模型在车辆计数的准确性和模型泛化能力方面取得了显著的进步。通过设计消融实验验证模型设计的合理性,与当前领先的计数方法进行比较验证模型性能。引入了焦点损失来解决类别分布不平衡的问题,在小型货车和大型货车的计数结果上取得了显著的提升。本文的研究成果不仅在车辆计数领域具有重要的应用价值,对于其他涉及到跨分辨率图像以及利用低分辨率图像的问题上,该模型也具有很强的迁移潜力。

